%----------------------------------------------------------------------------------------
% PACKAGES AND OTHER DOCUMENT CONFIGURATIONS
%----------------------------------------------------------------------------------------

\documentclass[paper=a4, fontsize=11pt]{scrartcl} % A4 paper and 11pt font size

\usepackage[margin=1in]{geometry}
\usepackage[T1]{fontenc} % Use 8-bit encoding that has 256 glyphs
\usepackage{fourier} % Use the Adobe Utopia font for the document - comment this line to return to the LaTeX default
\usepackage[english]{babel} % English language/hyphenation
\usepackage[utf8]{inputenc}
\usepackage{amsmath,amsfonts,amsthm} % Math packages
\usepackage{pgfplotstable}
\usepackage{pgfplots}
\usepackage{booktabs}
\usepackage{sectsty} % Allows customizing section commands
\allsectionsfont{\centering \normalfont\scshape} % Make all sections centered, the default font and small caps

\usepackage{fancyhdr} % Custom headers and footers
\pagestyle{fancyplain} % Makes all pages in the document conform to the custom headers and footers
\fancyhead{} % No page header - if you want one, create it in the same way as the footers below
\fancyfoot[L]{} % Empty left footer
\fancyfoot[C]{} % Empty center footer
\fancyfoot[R]{\thepage} % Page numbering for right footer
\renewcommand{\headrulewidth}{0pt} % Remove header underlines
\renewcommand{\footrulewidth}{0pt} % Remove footer underlines
\setlength{\headheight}{13.6pt} % Customize the height of the header

\numberwithin{equation}{section} % Number equations within sections (i.e. 1.1, 1.2, 2.1, 2.2 instead of 1, 2, 3, 4)
\numberwithin{figure}{section} % Number figures within sections (i.e. 1.1, 1.2, 2.1, 2.2 instead of 1, 2, 3, 4)
\numberwithin{table}{section} % Number tables within sections (i.e. 1.1, 1.2, 2.1, 2.2 instead of 1, 2, 3, 4)

\setlength\parindent{0pt} % Removes all indentation from paragraphs - comment this line for an assignment with lots of text

%----------------------------------------------------------------------------------------
% TITLE SECTION
%----------------------------------------------------------------------------------------

\newcommand{\horrule}[1]{\rule{\linewidth}{#1}} % Create horizontal rule command with 1 argument of height

\title{
\normalfont \normalsize
\textsc{Hochschule für Angewandte Wissenschaften Hamburg, Department Informatik} \\ [25pt] % Your university, school and/or department name(s)
\horrule{0.5pt} \\[0.4cm] % Thin top horizontal rule
\huge Algorithmen und Datenstrukturen - Aufgabe 2 \\ % The assignment title
\horrule{2pt} \\[0.5cm] % Thick bottom horizontal rule
}

\author{Team 3 Gruppe 4: Sönke Peters, Nils Eggebrecht, Markus Blechschmidt} % Your name

\date{\normalsize\today} % Today's date or a custom date

\begin{document}


\maketitle % Print the title

\subsubsection*{Abstract}
Durch die Aufgabe Nummer 2 in Algorithmen und Datenstrukturen soll uns die
Thematik der Komplexit\"at n\"aher gebracht werden. Dies geschieht am Beispiel vom
Alorithmen zum Finden und Testen von Primzahlen. Dieses Dokument behandelt die
Auswertung der Implementierungen.

\renewcommand{\contentsname}{Inhaltsangabe}
\renewcommand*\listtablename{Tabellen}
\renewcommand*\listfigurename{Darstellungen}
\tableofcontents
\listoftables
\listoffigures

\section{Theoretische Betrachtung}

In dieser Aufgabe sollen Algorithmen zum Suchen und bestimmen von Primzahlen
entsprechenden des Skripts implementiert und, wenn m\"oglich, optimiert werden.
Die Primzahlensuchen sollen \"uber Felder von Wahrheitswerten erfolgen, wobei der
Index die Zahlen repr\"asentiert. Die Problemgr\"o{\ss}e $N$ entspricht hierbei
der L\"ange des Feldes bzw. bei der Feststellung der Primzahleneigenschaft dem Wert der Zahl.

\subsection{Langsames Primzahl Suchen}\label{secSlow}
Bei dieser Implementation werden alle ganzen Zahlen von $2$ bis $N$ darauf
gepr\"uft, ob sie durch alle anderen ganzen Zahlen des gleichen Intervals teilbar
sind. Die Teilbarkeit wird durch die Modulofunktion bestimmt. Ist die zu testende
Zahl durch eine der anderen Zahlen teilbar, so wird sie aus dem Feld gestrichen.

Da dieser Algorithmus f\"ur jede Zahl des Bereiches den gesamten Bereich durchl\"auft,
erh\"alt man theoretisch einen Aufwand, welcher relativ zur Problemgr\"o{\ss}e
quadratisch w\"achst.

\subsection{Schnelles Primzahl Suchen}\label{secFast}
F\"ur diese Implementation wird bei jeder Zahl des Bereiches zum Testen, ob sie
eine Primzahl ist, die Funktion verwendet, die f\"ur \ref{secCheck} implementiert
wurde. Der Algorithmus aus \ref{secCheck} weist eine Aufwand auf, welcher sich
quadratwurzelartig zur Problemgr\"o{\ss}e verh\"alt. Da dieser in einer Schleife
auf jede Zahl des Bereiches angewandt wirdt, erh\"alt man, analog zu der Rechnung
auf Seite 29 des Skripst, ein Aufwandswachstum, welches proportional zu $N \cdot ln(N)$ ist.

\subsection{Sieb des Eratosthenes}\label{secEras}
Das Sieb des Eratosthenes wurde des Skripts entsprechend implementiert. Man geht
von der ersten Primzahl $2$ aus und streicht all ihre Vielfachen aus dem Feld. Danach
geht man zur n\"achsten noch nicht gestrichenen Zahl und wiederholt das Streichen der
Vielfachen. Dies wiederholt man so lange, bis man die Zahl \"uberschritten hat, die
der Quadratwurzel der Feldgr\"o{\ss}e entspricht.

\subsection{Primzahleigenschaft Feststellen}\label{secCheck}
Dieser Algorithmus pr\"uft die Primzahleigenschaft einer nat\"urlichen Zahl, indem
ihre Teilbarkeit durch alle nat\"urlichen Zahlen von $2$ bis zu ihrer Quadratwurzel
getestet wird. Dies basiert darauf, dass, falls es einen
weiteren Teiler gr\"o{\ss}er als die Quadratwurzel gibt, es auch einen geben muss,
der kleiner als diese ist. Laufzeit der funktion ist proportional zu $\sqrt{N}$ bzw.
besser, insofern man die suchenden Schleife abbricht, sobald ein Teiler gefunden wurde.

%-------------------------------

\section{Heuristische Auswertung}

\subsection{Primzahl Suche}

In der Tabelle \ref{table:1} sind Laufzeiten $T(N)$ unserer Implementierungen der
Primzahlensuchalgorithmen in Abh\"angigkeit des Problemgr\"o{\ss}eparamters $N$
zu sehen. Es wurde sich dazu entschieden, die Problemgr\"o{\ss}e exponentiell
ansteigen zu lassen, um einen breiten \"Uberblick der der Auswirkung zu bekommen.

In der Tabelle \ref{table:1} ist es bereits an  der Anzahl der Ziffern zu erkennen,
dass die erste Implementation einen deutlich h\"oheren Aufwand hat als die anderen
beiden. Weiterhin kann man auf Grund der Daten auch die Formel der Funktion f\"ur
den Aufwand der ersten Funktion erstellen. Diese lautet: $T(N)_{langsam} = N^2 -N +2$.
Die einzigen Ausnahem dazu sind die ersten beiden Messwerte, was wohl daran liegt,
dass der Algorithmus bei deren Werten sofort abbricht bzw. abk\"urzt.

Der Graph \ref{figure:1} stellt die Abh\"angigkeit zwischen Problemgr\"o{\ss}e und
Aufwand der ersten Implementation in einem lin/lin-Plot dar. Auch hier ist die
quadratische Abh\"agigkeit gut an der Parabelform des Graphen zu erkennen.

Graphen \ref{figure:2} und \ref{figure:3} plotten eine gleiche Grafik f\"ur die
anderen beiden Implementationen. Die Form von \ref{figure:2} best\"arkt die Vermutung
der Formel aus der \ref{secFast}. Die Form von \ref{figure:2} l\"asst auf einen
linearen Zusammenhang schlie{\ss}en.

\subsection{Primzahleigenschaft}

Sowohl die Werte aus \ref{table:2} als auch der Graph \ref{figure:4} zeigen gut
den quadratwurzelartigen Zusammenhang zwischen der Problemgr\"o{\ss}e und dem Aufwand.
Dies sieht man vor allem an der nach rechts ge\"offneten Parabel in \ref{figure:4}.

\begin{table}[h]
  \centering
  \pgfplotstabletypeset[
      column type=r,
      columns/N/.style={
        column name=$N$,
        /pgf/number format/fixed,
      },
      columns/slow/.style={
        column name=$T(N)_{langsam}$,
        /pgf/number format/fixed,
      },
      columns/fast/.style={
        column name=$T(N)_{schnell}$,
        /pgf/number format/fixed,
      },
      columns/eras/.style={
        column name=$T(N)_{Sieb}$,
        /pgf/number format/fixed,
      },
      every head row/.style={after row=\midrule},
  ]{search.dat}
  \caption{Tabelle des Aufwands $T(N)$ der verschiedenen Suchalgorithmen bei
  steigender Problemgroesse $N$}
  \label{table:1}
\end{table}

\begin{table}[h]
  \centering
  \pgfplotstabletypeset[
      column type=r,
      columns/N/.style={
        column name=$N$,
        /pgf/number format/fixed,
      },
      columns/slow/.style={
        column name=$T(N)_{check}$,
        /pgf/number format/fixed,
      },
      every head row/.style={after row=\midrule},
  ]{check.dat}
  \caption{Tabelle des Aufwands $T(N)$ des Algorithmus zum Pruefen der
  Primzahleigenschaft bei steigender Problemgroesse $N$}
  \label{table:2}
\end{table}

\begin{figure}[h]
  \centering
  \begin{tikzpicture}
    \begin{axis}[
      xlabel=N Problemgroesse,
      ylabel=T(N) Aufwand]
      \addplot table [x=N, y=slow]{search.dat};
    \end{axis}
  \end{tikzpicture}
  \caption{Graph des Aufwands bei der langsamen Suche}
  \label{figure:1}
\end{figure}

\begin{figure}[h]
  \centering
  \begin{tikzpicture}
    \begin{axis}[
      xlabel=N Problemgroesse,
      ylabel=T(N) Aufwand]
      \addplot table [x=N, y=fast]{search.dat};
    \end{axis}
  \end{tikzpicture}
  \caption{Graph des Aufwands bei der schnellen Suche}
  \label{figure:2}
\end{figure}

\begin{figure}[h]
  \centering
  \begin{tikzpicture}
    \begin{axis}[
      xlabel=N Problemgroesse,
      ylabel=T(N) Aufwand]
      \addplot table [x=N, y=eras]{search.dat};
    \end{axis}
  \end{tikzpicture}
  \caption{Graph des Aufwands bei dem Sieb des Eratosthenes}
  \label{figure:3}
\end{figure}

\begin{figure}[h]
  \centering
  \begin{tikzpicture}
    \begin{axis}[
      xlabel=N Problemgroesse,
      ylabel=T(N) Aufwand]
      \addplot table [x=N, y=check]{check.dat};
    \end{axis}
  \end{tikzpicture}
  \caption{Graph des Aufwands bei der Pruefen der Primzahleigenschaft}
  \label{figure:4}
\end{figure}

\end{document}
