%----------------------------------------------------------------------------------------
% PACKAGES AND OTHER DOCUMENT CONFIGURATIONS
%----------------------------------------------------------------------------------------

\documentclass[paper=a4, fontsize=11pt]{scrartcl} % A4 paper and 11pt font size

\usepackage[margin=2cm]{geometry}
\usepackage[T1]{fontenc} % Use 8-bit encoding that has 256 glyphs
\usepackage{fourier} % Use the Adobe Utopia font for the document - comment this line to return to the LaTeX default
\usepackage[english]{babel} % English language/hyphenation
\usepackage[utf8]{inputenc}
\usepackage{amsmath,amsfonts,amsthm} % Math packages
\usepackage{pgfplotstable}
\usepackage{pgfplots}
\usepackage{booktabs}
\usepackage{sectsty} % Allows customizing section commands
\allsectionsfont{\centering \normalfont\scshape} % Make all sections centered, the default font and small caps

\usepackage{fancyhdr} % Custom headers and footers
\pagestyle{fancyplain} % Makes all pages in the document conform to the custom headers and footers
\fancyhead{} % No page header - if you want one, create it in the same way as the footers below
\fancyfoot[L]{} % Empty left footer
\fancyfoot[C]{} % Empty center footer
\fancyfoot[R]{\thepage} % Page numbering for right footer
\renewcommand{\headrulewidth}{0pt} % Remove header underlines
\renewcommand{\footrulewidth}{0pt} % Remove footer underlines
\setlength{\headheight}{13.6pt} % Customize the height of the header

\numberwithin{equation}{section} % Number equations within sections (i.e. 1.1, 1.2, 2.1, 2.2 instead of 1, 2, 3, 4)
\numberwithin{figure}{section} % Number figures within sections (i.e. 1.1, 1.2, 2.1, 2.2 instead of 1, 2, 3, 4)
\numberwithin{table}{section} % Number tables within sections (i.e. 1.1, 1.2, 2.1, 2.2 instead of 1, 2, 3, 4)

\setlength\parindent{0pt} % Removes all indentation from paragraphs - comment this line for an assignment with lots of text

%----------------------------------------------------------------------------------------
% TITLE SECTION
%----------------------------------------------------------------------------------------

\newcommand{\horrule}[1]{\rule{\linewidth}{#1}} % Create horizontal rule command with 1 argument of height

\title{
\normalfont \normalsize
\textsc{Hochschule für Angewandte Wissenschaften Hamburg, Department Informatik} \\ [25pt] % Your university, school and/or department name(s)
\horrule{0.5pt} \\[0.4cm] % Thin top horizontal rule
\huge Algorithmen und Datenstrukturen - Aufgabe 2 \\ % The assignment title
\horrule{2pt} \\[0.5cm] % Thick bottom horizontal rule
}

\author{Team 3 Gruppe 4: Sönke Peters, Nils Eggebrecht, Markus Blechschmidt} % Your name

\date{\normalsize\today} % Today's date or a custom date

\begin{document}


\maketitle % Print the title

\subsubsection*{Abstract}
Durch die Aufgane Nummer 2 in Algorithmen und Datenstrukturen soll uns die
Thematik der Komplexitaet naeher gebracht werden. Dies geschieht am Beispiel vom
Alorithmen zum Finden und Testen von Primzahlen. Dieses Dokument behandelt die
Auswertung der Implementierungen.

\renewcommand{\contentsname}{Inhaltsangabe}
\renewcommand*\listtablename{Tabellen}
\renewcommand*\listfigurename{Darstellungen}
\tableofcontents
\listoftables
\listoffigures

\section{Theoretische Betrachtung}

In dieser Aufgabe sollen Algorithmen zum Suchen und bestimmen von Primzahlen
entsprechenden des Skripts implementiert und, wenn m\"oglich, optimiert werden.
Die Primzahlensuchen sollen \"uber Felder von Wahrheitswerten erfolgen, wobei der
Index die Zahlen repr\"asentiert. Die Problemgr\"o{\ss}e $N$ entspricht hierbei
der L\"ange des Feldes bzw. bei \ref{secCheck} dem Wert der Zahl.

\subsection{Langsames Primzahl Suchen}\label{secSlow}
Bei dieser Implementation werden alle ganzen Zahlen von $2$ bis $N$ darauf
gepr\"uft, ob sie durch alle anderen ganzen Zahlen des gleichen Intervals teilbar
sind. Die Teilbarkeit wird durch die Modulofunktion bestimmt. Ist die zu testende
Zahl durch eine der anderen Zahlen teilbar, so wird sie aus dem Feld gestrichen.

Da dieser Algorithmus f\"ur jede Zahl des bereiches den gesamten Bereicht durchl\"auft,
erh\"alt man theoretisch einen Aufwand, welcher relativ zur Problemgr\"o{\ss}e
quadratisch w\"achst.

\subsection{Schnelles Primzahl Suchen}\label{secFast}
F\"ur diese Implementation wird bei jeder Zahl des Bereiches zum Testen, ob sie
eine Primzahl sind, die Funktion verwendet, die f\"ur \ref{secCheck} implementiert
wurde. Da der Algorithmus aus \ref{secCheck} eine Aufwand aufweist, welcher sich
quadratwurzelartig zur Problemgr\"o{\ss}e verh\"alt, und dieser in einer Schleife
auf jede Zahl des Bereiches angewandt wirdt, erh\"alt man ein Aufwandswachstum,
welches proportional zu $N^{1.5}$ ist.

\subsection{Sieb des Eratosthenes}\label{secEras}

\subsection{Primzahleigenschaft Feststellen}\label{secCheck}

%-------------------------------

\section{Heuristische Auswertung}

\begin{table}[h]
  \centering
  \pgfplotstabletypeset[
      column type=r,
      columns/N/.style={
        column name=$N$,
        /pgf/number format/fixed,
      },
      columns/slow/.style={
        column name=$T(N)_{langsam}$,
        /pgf/number format/fixed,
      },
      columns/fast/.style={
        column name=$T(N)_{schnell}$,
        /pgf/number format/fixed,
      },
      columns/eras/.style={
        column name=$T(N)_{Sieb}$,
        /pgf/number format/fixed,
      },
      every head row/.style={after row=\midrule},
  ]{search.dat}
  \caption{Tabelle des Aufwands $T(N)$ der verschiedenen Algorithmen bei
  steigender Problemgroesse $N$}
  \label{table:1}
\end{table}

\begin{figure}[h]
  \centering
  \begin{tikzpicture}
    \begin{axis}[
      xlabel=N Problemgroesse,
      ylabel=T(N) Aufwand]
      \addplot table [x=N, y=slow]{search.dat};
    \end{axis}
  \end{tikzpicture}
  \caption{Graph des Aufwands bei der langsamen Suche}
  \label{figure:1}
\end{figure}

\begin{figure}[h]
  \centering
  \begin{tikzpicture}
    \begin{axis}[
      xlabel=N Problemgroesse,
      ylabel=T(N) Aufwand]
      \addplot table [x=N, y=fast]{search.dat};
    \end{axis}
  \end{tikzpicture}
  \caption{Graph des Aufwands bei der schnellen Suche}
  \label{figure:2}
\end{figure}

\begin{figure}[h]
  \centering
  \begin{tikzpicture}
    \begin{axis}[
      xlabel=N Problemgroesse,
      ylabel=T(N) Aufwand]
      \addplot table [x=N, y=eras]{search.dat};
    \end{axis}
  \end{tikzpicture}
  \caption{Graph des Aufwands bei dem Sieb des Eratosthenes}
  \label{figure:3}
\end{figure}

\begin{table}[h]
  \centering
  \pgfplotstabletypeset[
      column type=r,
      columns/N/.style={
        column name=$N$,
        /pgf/number format/fixed,
      },
      columns/slow/.style={
        column name=$T(N)_{check}$,
        /pgf/number format/fixed,
      },
      every head row/.style={after row=\midrule},
  ]{check.dat}
  \caption{Tabelle des Aufwands $T(N)$ des Algorithmus zum Pruefen der
  Primzahleigenschaft bei steigender Problemgroesse $N$}
  \label{table:1}
\end{table}


\begin{figure}[h]
  \centering
  \begin{tikzpicture}
    \begin{axis}[
      xlabel=N Problemgroesse,
      ylabel=T(N) Aufwand]
      \addplot table [x=N, y=check]{check.dat};
    \end{axis}
  \end{tikzpicture}
  \caption{Graph des Aufwands bei der Pruefen der Primzahleigenschaft}
  \label{figure:4}
\end{figure}

\end{document}
