%----------------------------------------------------------------------------------------
% PACKAGES AND OTHER DOCUMENT CONFIGURATIONS
%----------------------------------------------------------------------------------------

\documentclass[paper=a4, fontsize=11pt]{scrartcl} % A4 paper and 11pt font size

\usepackage[T1]{fontenc} % Use 8-bit encoding that has 256 glyphs
\usepackage{fourier} % Use the Adobe Utopia font for the document - comment this line to return to the LaTeX default
\usepackage[english]{babel} % English language/hyphenation
\usepackage[utf8]{inputenc}
\usepackage{amsmath,amsfonts,amsthm} % Math packages
\usepackage{pgfplotstable}
\usepackage{pgfplots}


\usepackage{sectsty} % Allows customizing section commands
\allsectionsfont{\centering \normalfont\scshape} % Make all sections centered, the default font and small caps

\usepackage{fancyhdr} % Custom headers and footers
\pagestyle{fancyplain} % Makes all pages in the document conform to the custom headers and footers
\fancyhead{} % No page header - if you want one, create it in the same way as the footers below
\fancyfoot[L]{} % Empty left footer
\fancyfoot[C]{} % Empty center footer
\fancyfoot[R]{\thepage} % Page numbering for right footer
\renewcommand{\headrulewidth}{0pt} % Remove header underlines
\renewcommand{\footrulewidth}{0pt} % Remove footer underlines
\setlength{\headheight}{13.6pt} % Customize the height of the header

\numberwithin{equation}{section} % Number equations within sections (i.e. 1.1, 1.2, 2.1, 2.2 instead of 1, 2, 3, 4)
\numberwithin{figure}{section} % Number figures within sections (i.e. 1.1, 1.2, 2.1, 2.2 instead of 1, 2, 3, 4)
\numberwithin{table}{section} % Number tables within sections (i.e. 1.1, 1.2, 2.1, 2.2 instead of 1, 2, 3, 4)

\setlength\parindent{0pt} % Removes all indentation from paragraphs - comment this line for an assignment with lots of text

%----------------------------------------------------------------------------------------
% TITLE SECTION
%----------------------------------------------------------------------------------------

\newcommand{\horrule}[1]{\rule{\linewidth}{#1}} % Create horizontal rule command with 1 argument of height

\title{
\normalfont \normalsize
\textsc{Hochschule für Angewandte Wissenschaften Hamburg, Department Informatik} \\ [25pt] % Your university, school and/or department name(s)
\horrule{0.5pt} \\[0.4cm] % Thin top horizontal rule
\huge Algorithmen und Datenstrukturen - Aufgabe 2 \\ % The assignment title
\horrule{2pt} \\[0.5cm] % Thick bottom horizontal rule
}

\author{Team 3 Gruppe 4: Sönke Peters, Nils Eggebrecht, Markus Blechschmidt} % Your name

\date{\normalsize\today} % Today's date or a custom date

\begin{document}


\maketitle % Print the title

\subsubsection*{Abstract}
Durch die Aufgane Nummer 2 in Algorithmen und Datenstrukturen soll uns die
Thematik der Komplexitaet naeher gebracht werden. Dies geschieht am Beispiel vom
Alorithmen zum Finden und Testen von Primzahlen. Dieses Dokument behandelt die
Auswertung der Implementierungen.

\renewcommand{\contentsname}{Inhaltsangabe}
\tableofcontents
\listoftables
\listoffigures

\section{Theoretische Betrachtung}

\subsection{Langsames Primzahl Suchen}

\subsection{Schnelles Primzahl Suchen}

\subsection{Sieb des Eratosthenes}

\subsection{Primzahleigenschaft Feststellen}

%-------------------------------

\section{Heuristische Auswertung}

\begin{table}[h!]
  \parbox{.5\linewidth}{
  \centering
  \begin{tabular}{ r | r }
    N & T(N) \\
    \hline
    1 & 0 \\
    2 & 0 \\
    4 & 14 \\
    8 & 58 \\
    16 & 242 \\
    32 & 994 \\
    64 & 4034 \\
    128 & 16258 \\
    256 & 65282 \\
    512 & 261634 \\
    1024 & 1047554 \\
    2048 & 4192258 \\
    4096 & 16773122 \\
    8192 & 67100674 \\
    16384 & 268419074 \\
    32768 & 1073709058 \\
  \end{tabular}
  \caption{Langsames Primzahl Suchen}
  \label{table:1}
  }
  \parbox{.5\linewidth}{
  \centering
  \begin{tabular}{ r | r }
    N & T(N) \\
    \hline
    1 & 0 \\
    2 & 0 \\
    4 & 10 \\
    8 & 22 \\
    16 & 50 \\
    32 & 115 \\
    64 & 259 \\
    128 & 594 \\
    256 & 1371 \\
    512 & 3200 \\
    1024 & 7521 \\
    2048 & 17918 \\
    4096 & 43359 \\
    8192 & 105994 \\
    16384 & 262725 \\
    32768 & 656624 \\
  \end{tabular}
  \caption{Schnelles Primzahl Suchen}
  \label{table:2}
  }
  \parbox{.5\linewidth}{
    \centering
    \begin{tabular}{ r | r }
      N & T(N) \\
      \hline
      1 & 0 \\
      2 & 0 \\
      4 & 8 \\
      8 & 19 \\
      16 & 44 \\
      32 & 96 \\
      64 & 203 \\
      128 & 420 \\
      256 & 861 \\
      512 & 1778 \\
      1024 & 3665 \\
      2048 & 7478 \\
      4096 & 15249 \\
      8192 & 31122 \\
      16384 & 63284 \\
      32768 & 128849 \\
    \end{tabular}
    \caption{Sieb des Eratosthenes}
    \label{table:3}
  }
  \parbox{.5\linewidth}{
    \centering
    \begin{tabular}{ r | r }
      N & T(N) \\
      \hline
      3 & 0 \\
      7 & 1 \\
      13 & 2 \\
      31 & 4 \\
      61 & 6 \\
      127 & 10 \\
      251 & 14 \\
      509 & 21 \\
      1021 & 30 \\
      2039 & 44 \\
      4093 & 62 \\
      8191 & 89 \\
      16381 & 126 \\
      32719 & 179 \\
    \end{tabular}
    \caption{Primzahleigenschaft Feststellen}
    \label{table:4}
  }
\end{table}

\begin{tikzpicture}
\begin{axis}[
  xlabel=N Problemgroesse,
  ylabel=T(N) Aufwand]
\addplot table [x=N, y=slow]{data.dat};
\addlegendentry{langsame Suche}
\end{axis}
\end{tikzpicture}

\begin{tikzpicture}
\begin{axis}[
  xlabel=N Problemgroesse,
  ylabel=T(N) Aufwand]
\addplot table [x=N, y=fast]{data.dat};
\addlegendentry{schnelle Suche}
\addplot table [x=N, y=eras]{data.dat};
\addlegendentry{Sieb des Eratosthenes}
\end{axis}
\end{tikzpicture}

\end{document}
